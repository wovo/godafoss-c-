\documentclass{article}

\author{Wouter van Ooijen}
\title{Programming micro-controllers with godafoss}
\date{\today}

\setlength{\parindent}{0em}
\setlength{\parskip}{1em}

\usepackage{listings}
\usepackage{xcolor}
\lstset { %
    language=C++,
    backgroundcolor=\color{black!5}, % set backgroundcolor
    basicstyle=\footnotesize,% basic font setting
}

\begin{document}

\maketitle

% =======================================================
\section{Introduction}

\begin{itemize}
\item  target system
\item  target micro-controller
\item  download
\item  compile/build versus run
\item  C++ versus godafoss
\item  electronics 'on the fly'
\item  format, whitespace
\end{itemize}

% =======================================================
\section{First steps}

% ---------------------------------------------------------------------------
\subsection{Blink}

The first thing to do on micro-controller system is to blink a LED.
It proves that your system is working and that you are a genius.
So let's look at the code for a blink-a-led application in godafoss.
The line numbers are for reference, they are not part of the real
source code.

% update quote( input, "due-01-01-01-blink-led", "", numbers = 1 ) 
\lstset {language=C++}
\begin{lstlisting}
[ 1] #include "godafoss.hpp"
[ 2] 
[ 3] int main(){
[ 4]    godafoss::blink< 
[ 5]       godafoss::target<>::led, 
[ 6]       godafoss::target<>::timing::ms< 500 > 
[ 7]    >();
[ 8] }
\end{lstlisting}

The first line makes the godafoss features available for use in the code.
It must be on a line by itself, and it is the first line of every godafoss
application.

The int main(){ on line 3 upto the matching } on line 8 are required
by the C++ language to identify what the application does when it is run.
You will find such a 'main function' in every C++ application. 
The int before main is required by the C++ language because it is 
useful in C++ applications that run on a PC.
For a micro-controller application it has no meaning, but it must still be 
present.
Just pretend that it is not there.

The lines 4 .. 7 are the interesting part.
The godafoss::blink phrase means 'the blink within the godafoss namespace'. 
A namsepace is a bit like a family.
There are many persons named Martin in the world, so to identify a specific
Martin you say something like Martin King: the person named Martin
within the King family. 
(Assuming there is only one King family, and only one
Martin in that family.)
In C++ you would write this as King::Martin.
The godafoss namespace
has a blink function, which is used here to do the blinking.

We want to blink the LED that is on the target board.
This LED is connected to a specific pin of the micro-controller
on which the application will run.
To have the desired effect, the blink function must be configured to 
do the blinking on the correct micro-controller pin. 
The godafoss namespace has a target<> in it that contains the 
features available on the target system.
For each target that has an LED, godafoss provides this pin
as godafoss::target<>::led, so this is the first item used
to configure the blink function.
The angle brackets <> after target are required because a target 
can be configured.
Specifying <> means that the default(s) are used.

The next thing that must be configured is how long a blink period
(the sum of the on and off times) must be.
This is specified via the timing service that is provided by each godafoss target.
In this example 500 ms (half a second) is specified 
(to be divided evenly among the on and the off period), 
so the blinking will be at a rate of 2 per second (2 Hertz).

The fully configured godafoss::blink doesn't need any run-time information 
to do its work, so the () on line 7 specifies that it is called,
which means that its code is executed.
As configured godafoss::blink will blink forever, so that
is all there is to blinking the LED on a target board.

The next step is to blink an LED that is external to the board.
This requires the LED to be connected properly, and 
the blink function must be configured with the micro-controller
pin that it is connected to. 
A suitable series resistor is needed to limit the current,
otherwise the LED or the micro-controller might be damaged.
In nearly all cases, a 1 kOhm (1000 Ohm) resistor will do.
Such a resistor has a color code of brown-black-red 
or brown-black-black-brown,
in both cases followed by other rings that are not important to us.

image

A resistor is symmetrical (apart from aesthetics you can swap its two leads).
An LED however is not: it must be connected the right way round, 
or it will not work.

image

Use a solderless breadboard and plug-in wires to connect 
the pin marked 2 or D2 to the resistor, 
the other side of the resistor to the LED pin at the round side
of the LED, and the other (flat) side of the LED to the ground 
(marked GND on most boards).

image
image

Provided that the correct Arduino target board is selected in the 
development environment, the godafoss::target namsepace will contain
the pins as d0, d1, etc and a0, a1, etc.
The LED is connected to the d7 pin, so that pin must be used
to configure the blink function.

% update quote( input, "due-01-01-02-blink-d2", "", numbers = 1 )
\lstset {language=C++}
\begin{lstlisting}
[ 1] #include "godafoss.hpp"
[ 2] 
[ 3] int main(){
[ 4]    godafoss::blink< 
[ 5]       godafoss::target<>::d2, 
[ 6]       godafoss::target<>::timing::ms< 500 > 
[ 7]    >();
[ 8] }
\end{lstlisting}

All this godafoss and godafoss::target is a bit longwieldy and repetitive.
Aliases can be created via 'using' statements to make the application 
code itself (the main) shorter and easier to read, at the cost of 
a few extra lines at the top.

% update quote( input, "due-01-01-03-blink-using", "", numbers = 1 )
\lstset {language=C++}
\begin{lstlisting}
[ 1] #include "godafoss.hpp"
[ 2] 
[ 3] using gf     = godafoss;
[ 4] using target = gf::target<>;
[ 5] using timing = gf::timing;
[ 6] 
[ 7] int main(){
[ 8]    gf::blink< 
[ 9]       target::d2, 
[10]       timing::ms< 500 > 
[11]    >();
[12] }
\end{lstlisting}

One step further, it is often considered good practice to separate
the 'flexible' (easily changeable) part of the application from 
the details of how how it works.
In the case of the blink application, the pin that is used and the
blinking period can be lifted out of the blink call.

% update quote( input, "due-01-01-04-blink-using-led-period", "", numbers = 1 ) -->
\lstset {language=C++}
\begin{lstlisting}
[ 1] #include "godafoss.hpp"
[ 2] 
[ 3] using gf     = godafoss;
[ 4] using target = gf::target<>;
[ 5] using timing = gf::timing;
[ 6] 
[ 7] using led    = target::d2;
[ 8] using period = timing::ms< 500 >; 
[ 9] 
[10] int main(){
[11]    gf::blink< led, period >();
[12] }
\end{lstlisting}

This separation might seem unnecessary for such a small and simple
program, but it pays off for a larger application,
where the led and the period are likely to be used at
multiple locations in the source.

% ---------------------------------------------------------------------------
\subsection{Morse}

Morse is a way to encode letters and digits in a sequence of 
short and long beeps (or flashes of light). 
A famous example is the sequence that was used as a distress
(call for help) signal: three short beeps, three longer beeps, 
three short beeps. 
Interpreted as morse code this can be read as SOS, 
which was in popular culture read as Save Our Ship or 
(later) Save Our Souls.
(Note: the real morse code for the letters SOS would be
three short beeps, *a pause*, three long beeps, *a pause*,
three short beeps. The pauses are needed to separate the letters.)

Beeping is much like blinking an LED, only faster.
The so-called middle A note, commonly used for tuning, has a frequency 
of 440 Hz, which means that the full period is 1/440 s = 2.27 ms.
Hence the configuration parameter for blink to produce a middle A
is half that value: 2270 us.

To physically produce a beep, a speaker must be connected to the target.
As for the LED, a series resistor must be used to limit the current.
Like a resistor, a speaker is symmetrical 
(swapping the connections has no effect).
A 1k resistor can be used to get a soft sound, 100 Ohm 
(brown-black-brown or brown-black-black-black)
will produce a louder sound.

image

Changing the 500 ms used in the blink applications to 2270 us produces
a continuous 440 Hz tone.

% update quote( input, "due-01-02-01-beep-440", "", numbers = 1 ) -->
\lstset {language=C++}
\begin{lstlisting}
[ 1] #include "godafoss.hpp"
[ 2] 
[ 3] using gf     = godafoss;
[ 4] using target = gf::target;
[ 5] using timing = gf::timing;
[ 6] 
[ 7] using led    = target::d3;
[ 8] using period = timing::us< 1120 >; 
[ 9] 
[10] int main(){
[11]    gf::blink< led, period >();
[12] }
\end{lstlisting}

image frequency meter 440 Hz

The blink function that was used so far blinks forever.
A second version of this function exists, that takes a second
duration parameter, to specify how long the blinking
(or in this case, the beeping) must be done.


- beep
- blink an amount of time
- different on and off periods

% =======================================================
\section{Scale factors}


% =======================================================
\section{Resistor color codes}

% =======================================================
\section{Arduino pin names}


% =======================================================
\section{Using a solderless breadboard}

\end{document}

