This reference document describes target and peripheral chips and modules
supported by the godafoss library.


target chips
target modules
peripheral chips
peripheral modules

teensy 4.1 :)

stm32f103c8


/// \brief
/// hwlib HAL for the stm32f103c8
///
/// This namespace contains the hwlib implementation of the pins, timing
/// and (software) UART output for the stm32f103c8.
///
/// Initially, the chip runs from its HSI internal RC oscillator at 8 MHz.
/// The first wait call configures the chip to run at 72 MHz,
/// assuming an 8 MHz crystal.
///
/// The chip runs at 3.3 Volt and that is the level on its IO pins.
///
/// References:
///    - <A HREF="http://www.st.com/content/ccc/resource/technical/document/reference_manual/59/b9/ba/7f/11/af/43/d5/CD00171190.pdf/files/CD00171190.pdf/jcr:content/translations/en.CD00171190.pdf">
///       RM0008 STM32F1xxx reference manual</A> (pdf)
///    - <A HREF="http://www.st.com/content/ccc/resource/technical/document/datasheet/33/d4/6f/1d/df/0b/4c/6d/CD00161566.pdf/files/CD00161566.pdf/jcr:content/translations/en.CD00161566.pdf">
///       STM32Fl03x8 datasheet</A> (pdf)
///

db103

/// \brief
/// hwlib HAL for the DB103 board
///
/// \image html db103-pcb.jpg
//
/// This namespace contains the hwlib implementation of the pins, timing
/// and (sotware) UART output for the DB103 board (LPC1114FN28/102 chip).
/// The chip runs at its default speed of 12 MHz,
/// from its internal RC oscillator.
///
/// The port and pin parameters to the constructors of the pin classes
/// refer to the LPC1114 ports and pins, as shown on the DB103 PCB.
///
/// The chip runs at 3.3 Volt and that is the level on its IO pins.
///
/// This implementation doesn't (yet) offer:
///    - hardware UART
///    - full CPU speed (48 MHz)
///    - hardware SPI
///
/// References:
///    - <A HREF="http://www.nxp.com/documents/user_manual/UM10398.pdf">
///       LPC1114 user manual</A> (pdf)
///

arduino due

/// \brief
/// hwlib HAL for the Arduino Due
///
/// \image html due-pcb.jpg
//
/// This namespace contains the hwlib implementation of the pins, timing
/// and (software) UART output for the Arduino Due (ATSAM3X8E chip).
/// The first wait call configures the chip to run at 84 MHz.
///
/// The port and pin parameters to the constructors of the pin classes
/// can either use to the ATSAM3X8E ports and pins, or the Arduino names.
///
/// The Due has an on-board orange LED connected to port 1 pin 27.
///
/// The chip runs at 3.3 Volt and that is the level on its IO pins.
///
/// \image html due-pinout.png
///
/// The ATSAM3X8E chip has a watchdog system that is enabled by default.
/// If left alone, the watchdog will reset the chip after a short time.
/// To prevent this, the watchdog is disabled on the first
/// timing call (wait_*() or now_*()).
///
/// References:
///    - <A HREF="https://www.arduino.cc/en/uploads/Main/arduino-uno-schematic.pdf">
///       Arduino Due circuit reference diagram</A> (pdf)
///    - <A HREF="http://www.atmel.com/ru/ru/Images/Atmel-11057-32-bit-Cortex-M3-Microcontroller-SAM3X-SAM3A_Datasheet.pdf">
///       ATSAM38XE datasheet</A> (pdf)
///


matrix keypad

max7219


pcf8591

/// pcf8591 I2C A/D and D/A converter
///
/// This class implements an interface to a pcf8591 A/D & D/A converter chip.
///
/// \image html pcf8591-pinout.png
///
/// A pcf8591 is an I2C slave that provides 4 8-bits A/D inputs (that can
/// be arranged in various combinations of single and balanced inputs),
/// and a single 8-bits D/A output.
/// The power supply range is 2.5 .. 5.5 Volt.
/// Of the 7-bit slave address,
/// 3 bits are set by the level of 3 input pins (A0..A2) of the chip.
/// When all 3 are pulled low, the slave address is 0x48.
///
/// The next code repeatedly prints the values read
/// by the 4 (single-ended) A/D converters.
///
/// \snippet "db103\pcf8591\main.cpp" [Doxygen pcf8591-adc example]
///
/// references:
///    - <A HREF="http://www.nxp.com/documents/data_sheet/PCF8591.pdf">
///       PCF8591 data sheet</A> (pdf)
///

pcf8574

/// pcf8574 / pcf8574a I2C I/O extender
///
/// This class implements an interface to a pcf8574 and pcf8574a
/// I2C I/O extender chip.
///
/// \image html pcf8574a-pinout.png
///
/// A pcf8574(a) is an I2C slave that provides 8 open-collector input/output
/// pins with weak pull-ups.
/// The power supply range is 2.5 .. 5.5 Volt.
/// Of the 7-bit slave address,
/// 3 bits are set by the level of 3 input pins (A0..A2) of the chip.
/// With all pulled low the iw2c address is 0x38.
///
/// \image html pcf8574a-iopin.png
///
/// The chip has only one register, which can be read and written.
/// When written, it determines the level of the 8 output pins:
/// low when the bit is 0, pulled weakly high when the bit is 1.
/// When read, the level of the 8 pins determines the value:
/// 0 for a low pin, 1 for a high pin.
///
/// \image html pcf8574a-diagram.png
///
/// The next code shows a kitt display
/// on 8 LEDs connected to the PCF8574A output pins.
/// Because the output pins are open-collector, the LEDs
/// are connected to power (instead of to the ground), hence
/// the use of hwlib::port_out_invert().
/// \snippet "db103\pcf8574a-blink\main.cpp" [Doxygen pcf8574a example]
///
/// The pcf8574 and pcf8574a are the same chips, but with a different
/// I2C bus address.
///
/// \image html pcf8574a-addresses.png
///
/// references:
///    - <A HREF="http://www.nxp.com/documents/data_sheet/PCF8574_PCF8574A.pdf">
///       PCF8574A data sheet</A> (pdf)
///

5510

/// Nokia 5510 B/W graphics LCD
///
/// This class implements an interface to the type of LCD
/// that was used in older Nokia telephones.
/// It is a 84 columns x 48 lines black-and-white LCD.
///
/// This type of LCD is cheap and available from lots of sources,
/// but the quality is often abominable.
///
/// \image html lcd5510-empty.jpg
///

oled

/// \brief
/// Oled B/W graphics LCD
/// \details
/// This class implements an interface to an 128 x 64 pixel
/// monochrome (on/off) OLED display. These displays are available
/// in various colcors (green, red, white, etc.).
/// The interface is I2C.
/// The driver chip is an SSD1306.
///
/// The interface is buffered: all writes
/// are buffered in memory until flush() is called.
///
/// When the PCB has regulator (3-legged component) the power can be 3 - 5V.
/// If it hasn't, you can use only 3.3V.
///
/// There are variations of this display with more pins, which
/// have more interface options (SPI as alternative interface).
///
/// This type of display is reasonably priced
/// and available from lots of sources.
///
/// \image html oled.jpg
//


hd44780

/// hd44780 character LCD interface
///
/// This class implements an interface to an hd44780 character LCD.
///
/// \image html hd44780-picture.jpg
///
/// The hd44780 is the standard chip for interfacing small dot-character
/// LCD interfaces.
/// It can display the ASCII table characters, 8 characters (0..7)
/// that can be user-defined as 5x7 pixels, and a an upper 128 characters
/// (128...255) that vary with the chip variant, often Japanese characters.
///
/// The chip and its digital pins run at 5V.
/// The contrast input can in most cases be connected to 0V (ground), but
/// better is to use a 10k potentiometer between 0V and 5V. Some
/// displays (mostly extended-temperature-range types) need the lower
/// size of this potentiometer tied to a negative voltage, for instance -5V.
///
/// The digital interface to the chip has 8 data lines, but the chip can be
/// configured to use only 4. This adds some complexity to the driver
/// software and slows it down a little, but the saving of 4 micro-controller
/// more than compensates for this, hence nearly all software
/// (including this driver) for is for the 4-bit interface.
/// Note for the 4-bit interface the 4 highest data pins (D4..D7) are used.
/// The lower 4 can be left unconnected.
///
/// The chip has some locations that can be writen and also read back,
/// but this offers little advantage, so most software (including this driver)
/// only writes to the chip, thus saving another pin.
/// Hence 6 pins (+ ground and 5V)
/// are needed to interface to an hd44780 display:
/// 4 data lines, the R/S line (selects between command and data),
/// and the E line (a strobe for the command).
///
/// \image html hd44780-connection.png
///
/// (Some larger displays use not one but two hd44780 chips.
/// This interface is not compatible with such LCDs.)
///
/// Most hd44780 LCDs have a single row of connections,
/// with the following pinout:
///
/// \image html hd44780-pinout.jpg
///
/// But as always, check the datasheet (in this case of the LCD) to be sure!
///
/// The hd44780 implements the ostream interface, but it doesn't scroll:
/// while the cursor is outside the visible characters (beyond the end of the line,
/// or beyond the number of lines) any character writes will be ignored.
/// Some characters are treated special:
///    - '\\n' clears the rest of the line, and then
///      moves to the first position of the next line
///    - '\\r' puts the cursor at the start of the current line
///    - '\\c' moves the cursor to the top-left position
///
/// The best way to get a flicker-free display is to overwrite
/// instead of clear-and-then-write:
/// use '\\c' to got to the 'origin', then rewrite the whole display,
/// using '\\n' to go to a next line (because it clears the remainder of the line).
///
/// references:
///    - <A HREF="https://en.wikipedia.org/wiki/Hitachi_HD44780_LCD_controller">
///       Hitachi HD44780 LCD controller</A> (wikipedia)
///    - <A HREF="https://www.sparkfun.com/datasheets/LCD/HD44780.pdf">
///       HD44780U data sheet</A> (pdf)
///

74hc595

/// hc595 8-bit output shift register
///
/// This class implements an interface to
/// an hc595 8-bit output shift register chip.
///
/// \image html hc595-pinout.png
///
/// The hc595 is an 8-bit serial-in parallel-out shift register,
/// connected to an 8-bit data storage register.
/// The 8 outputs of the data register are available on chip pins.
/// The power supply range is 2.0 .. 6.0 Volt.
///
/// \image html hc595-diagram.png
///
/// The HC595 can be used as SPI output-only peripheral:
///    - connect MR (active-low reset input) to the power
///    - connect OE (active-low output enable) to gound
///    - use DS as MOSI
///    - use SHCP as SLCK
///    - use STCP as SS
///
/// Note that STCP is not a real chip-select: the chip will always
/// respond to the data that is clocked by storing it in the shift register.
/// But only the chip that got the select signal will (on the rising edge
/// of the SS) transfer the data from the shift register to the storage
/// register, hence affecting the outputs.
///
/// The next code shows a kitt display
/// on 8 LEDs connected to the HC595 output pins:
/// \snippet "db103\hc595\main.cpp" [Doxygen hc595 example]
///
/// The 74HCT595 is a similar chip, but intended (only) for 5V power,
/// and for use with the old TTL signal levels
/// (HC chips are for the CMOS signal levels that are more common now).
///
/// references:
///    - <A HREF="https://www.nxp.com/documents/data_sheet/74HC_HCT595.pdf">
///       74HC595/74HCT595 data sheet</A> (nxp, pdf)
///
