\documentclass{article}

\author{Wouter van Ooijen}
\title{Programming micro-controllers with godafoss}
\date{\today}

\setlength{\parindent}{0em}
\setlength{\parskip}{1em}

\usepackage{listings}
\usepackage{xcolor}
\lstset { %
    language=C++,
    backgroundcolor=\color{black!5}, % set backgroundcolor
    basicstyle=\footnotesize,% basic font setting
}

\begin{document}

\maketitle

% ==============================================================================
\section{Introduction}

\begin{itemize}
\item  introductory text, not a reference
\item  target system
\item  target micro-controller
\item  download
\item  compile/build versus run
\item  C++ versus godafoss
\item  electronics 'on the fly'
\item  format, whitespace
\end{itemize}

% ==============================================================================
%
\section{First steps}
%
% ==============================================================================

% ------------------------------------------------------------------------------
\subsection{First blood: blink}

The first thing to do on micro-controller system is to blink a LED.
It proves that your system is working and that you are a genius.
So let's look at the code for a blink-a-led application in godafoss.
The line numbers are for reference, they are not part of the real
source code.

% update quote( input, "due-01-01-01-blink-led", "", numbers = 1 )
\lstset {language=C++}
\begin{lstlisting}
[ 1] #include "godafoss.hpp"
[ 2]
[ 3] int main(){
[ 4]    godafoss::blink<
[ 5]       godafoss::target<>::led,
[ 6]       godafoss::target<>::timing::ms< 500 >
[ 7]    >();
[ 8] }
\end{lstlisting}

The first line makes the godafoss features available for use in the code.
It must be on a line by itself, and it is the first line of every godafoss
application.

The "int main(){" on line 3 upto the matching "}" on line 8 are required
by the C++ language to identify what the application does when it is run.
You will find such a 'main function' in every C++ application.
The int before main is required by the C++ language because it is
useful in C++ applications that run on a PC.
For a micro-controller application it has no meaning, but it must still be
present.
Just pretend that it is not there.

The lines 4 .. 7 are the interesting part.
The "godafoss::blink" on line 4 phrase means
'the blink within the godafoss namespace'.
A namsepace is a bit like a family.
There are many persons named Martin in the world, so to identify a specific
Martin you say something like Martin King: the person named Martin
within the King family.
(Assuming there is only one King family, and only one
Martin in that family.)
In C++ you would write this as King::Martin.
The godafoss namespace has a blink function,
which is used here to do the blinking.

The things between the "<" on line 4 and the ">;" on line 7 are
fixed things we pass to blink to configure it for our purpose,
in this case the LED pin and the blink time.

We want to blink the LED that is on the target board.
This LED is connected to a specific pin of the micro-controller
on which the application will run.
To have the desired effect, the blink function must be configured to
do the blinking on the correct micro-controller pin.
The godafoss namespace has a target<> in it that contains the
features available on the target system.
For each target that has an LED, godafoss provides this pin
as godafoss::target<>::led, so this is the first item used
to configure the blink function.
The angle brackets <> after target are required because a target
can be configured.
Specifying <> means that the default(s) are used.

The next thing that must be configured is how long a blink period
(the sum of the on and off times) must be.
This is specified via the timing service that is provided by each godafoss target.
In this example 500 ms (half a second) is specified
(to be divided evenly among the on and the off period),
so the blinking will be at a rate of 2 per second (2 Hertz).

The fully configured godafoss::blink doesn't need any run-time information
to do its work, so the "();" on line 7 specifies that it is called,
which means that its code is executed.
As configured godafoss::blink will blink forever, so that
is all there is to blinking the LED on a target board.

The next step is to blink an LED that is external to the board.
This requires the LED to be connected properly, and
the blink function must be configured with the micro-controller
pin that it is connected to.
A suitable series resistor is needed to limit the current,
otherwise the LED or the micro-controller might be damaged.
In nearly all cases, a 1 kOhm (1000 Ohm) resistor will do.
Such a resistor has a color code of brown-black-red
or brown-black-black-brown,
in both cases followed by other rings that are not important to us.

image

A resistor is symmetrical (apart from aesthetics you can swap its two leads).
An LED however is not: it must be connected the right way round,
or it will not work.

image

Use a solderless breadboard and plug-in wires to connect
the pin marked 2 or D2 to the resistor,
the other side of the resistor to the LED pin at the round side
of the LED, and the other (flat) side of the LED to the ground
(marked GND on most boards).

image
image

Provided that the correct Arduino target board is selected in the
development environment, the godafoss::target namsepace will contain
the pins as d0, d1, etc and a0, a1, etc.
(For another target board the pins available in godfafoss::target<>
will likely be different.)
In this example LED is connected to the d7 pin,
so that pin must be used to configure the blink function.

% update quote( input, "due-01-01-02-blink-d2", "", numbers = 1 )
\lstset {language=C++}
\begin{lstlisting}
[ 1] #include "godafoss.hpp"
[ 2]
[ 3] int main(){
[ 4]    godafoss::blink<
[ 5]       godafoss::target<>::d2,
[ 6]       godafoss::target<>::timing::ms< 500 >
[ 7]    >();
[ 8] }
\end{lstlisting}

All this godafoss, godafoss::target and godafoss::target<>::timing
is a bit unwieldy and repetitive.
Aliases can be created via 'using' statements to make the application
code itself (the main) shorter and easier to read, at the cost of
a few extra lines at the top.

% update quote( input, "due-01-01-03-blink-using", "", numbers = 1 )
\lstset {language=C++}
\begin{lstlisting}
[ 1] #include "godafoss.hpp"
[ 2]
[ 3] using gf     = godafoss;
[ 4] using target = gf::target<>;
[ 5] using timing = gf::timing;
[ 6]
[ 7] int main(){
[ 8]    gf::blink<
[ 9]       target::d2,
[10]       timing::ms< 500 >
[11]    >();
[12] }
\end{lstlisting}

One step further, it is often considered good practice to separate
the 'flexible' (easily changeable) part of the application from
the details of how how it works.
In the case of the blink application, the pin that is used and the
blinking period can be lifted out of the blink call.

% update quote( input, "due-01-01-04-blink-using-led-period", "", numbers = 1 ) -->
\lstset {language=C++}
\begin{lstlisting}
[ 1] #include "godafoss.hpp"
[ 2]
[ 3] namespace gf  = godafoss;
[ 4] using target  = gf::target<>;
[ 5] using timing  = gf::timing;
[ 6]
[ 7] using led     = target::d2;
[ 8] using period  = timing::ms< 500 >;
[ 9]
[10] int main(){
[11]    gf::blink< led, period >();
[12] }
\end{lstlisting}

This separation might seem unnecessary for such a small and simple
program, but it pays off for a larger application,
where the led and the period are likely to be used at
multiple locations in the source.

% ------------------------------------------------------------------------------
\subsection{Timing : morse}

Morse is a way to encode letters and digits in a sequence of
short and long beeps (or flashes of light) and pauses.
A famous example is the sequence that was used as a distress
(call for help) signal: three short beeps, three longer beeps,
three short beeps.
Interpreted as morse code this can be read as SOS,
\footnote{The real morse code for the letters SOS would be
three short beeps, \textit{a pause}, three long beeps, \textit{a pause},
three short beeps. The pauses are needed to separate the letters.}
which was in popular culture read as Save Our Ship or
(later) Save Our Souls.

Beeping is much like blinking an LED, only faster.
The so-called middle A note, commonly used for tuning, has a frequency
of 440 Hz, which means that the full period is 1/440 s = 2.27 ms.
Hence the configuration parameter for blink to produce a middle A
is half that value: 2270 us.
\footnote{A millisecond or ms is 1/1000 of a second, a microsecond or us
is 1/1000 of a millisecond.}

To physically produce a beep, a speaker must be connected to the target.
As with the LED, a series resistor must be used to limit the current.
Like a resistor, a speaker is symmetrical
(swapping the connections has no effect).
A 1k resistor can be used to get a soft sound, 100 Ohm
(brown-black-brown or brown-black-black-black)
will produce a louder sound.

image

Changing the 500 ms used in the blink applications to 2270 us produces
a continuous 440 Hz tone.

% update quote( input, "due-01-02-01-beep-440", "", numbers = 1 ) -->
\lstset {language=C++}
\begin{lstlisting}
[ 1] #include "godafoss.hpp"
[ 2]
[ 3] namespace gf  = godafoss;
[ 4] using target  = gf::target;
[ 5] using timing  = gf::timing;
[ 6]
[ 7] using led     = target::d3;
[ 8] using period  = timing::us< 1120 >;
[ 9]
[10] int main(){
[11]    gf::blink< led, period >();
[12] }
\end{lstlisting}

image frequency meter 440 Hz

The blink function that was used so far blinks forever.
A second version of this function exists, that takes a second
duration parameter, to specify how long the blinking
(or in this case, the beeping) must last.


- beep
- blink an amount of time
- different on and off periods

% ------------------------------------------------------------------------------
\subsection{Kitt}

A Kitt display (from the once famous Knight Rider series) is a line
of LEDs, where the one LED that is on moves back and forth.
The next code does this manually with four LEDs connected to the pins
d2 .. d5.
\footnote{On most Arduino boards, the pins d0 and d1 are used
for the communication between the Arduino and the PC, so these pins
can't be used for other purposes.}



- how to write a kitt yourself
   - create

Each

subsection boxes, concepts, modifiers-constructors

text display

graphics

decorators

opt-in decorators

background functions

multithreading

C++ subset



% ==============================================================================

\section{Scale factors}


% ==============================================================================

\section{Resistor color codes}

% ==============================================================================

\section{Arduino pin names}


% ==============================================================================

\section{Geany}

% ==============================================================================

Geany is a text editor that can can be used to invoke an external makefile.
Some points to note:
- Build-> Make : call the make file, build is the default goal
- you will build in the dircetory of your current file,
  there is no notion of a current project
  - geany as editor, make build, make run (clean?)
- start geany with sudo
- run the excutable through a run script with setuid?

% ==============================================================================

\section{Using a solderless breadboard}

% ==============================================================================

\section{Code style and conventions}

- everything is in godafoss::
- if it can be a class/struct, make it so
- no blanket 'using' in the examples
- 1'000'000
- snake_case for everything (template parameters?)
- indentation is 3 spaces
- max line length is 80 characters
- 1TBS
- when vertical separator alignment, 2 spaces before
- prefer short files, a few pages at most in GreenPrint
- Boost license (for godafoss code, chip headers may vary)
- start with _underscore for private stuff (like Python)
- encode parameter order in name (this encourages shorter parameter lists!)
- macro's start with GODAFOSS_
- C++20, might switch to later version when available
- -Werror, enable as much warnings as reasonably possible
- be kind to a reader of the code
- overload names (keep the set of names limited)
- RAM is more scarce than ROM
- the user should not be able to trigger UB inside library code
- native support is for testing and debugging, not for developing applications
- windows support is MinGW64 only (problem with weak)
- DRY even if that requires a MACRO
- no heap, no exceptions, no RTTI,
- no code indirections, except for background and threading
- no floats unless told to do so by the user
- library files start with gf-, except for godafoss.hpp
- when contributing: add name and year to the Copyright notice
- application reader > application writer > library reader > library writer
- gf-all handles: std includes, external includes, warnings, namespace godafoss
- explicit private: / public:
- use by_const<> instead of by-value passing

naming and architecture
- no heap, no RTTI, no exceptions, no function indirection
- constexpr rulez! (provide alternative to math?)
- bikesheds are important, especially their color
- header-only, (only) include the target
- lowercase with _ separators
- name composition order is from outer to inner, then verb, then variations
- leading _ is library-internal (can change without notice)
- macro's moeten _lowercase zijn!
- avoid verb / noun / adjective confusion
- avoid negatives (no-, not-, un- etc)
- avoid similar things or couples with the same prefix (source / sink)
- a concept is a type, so it doesn't need an is_ prefix (you don't use is_int!)
- prefer composition that changes features over extra object features (direct<> instead of read_flush in the class)
- avoid using the name of what you are in, it might change (including the library name)
- short names for small scopes, longer name for wider scopes
- short names for common operations, longer names for less common / dangerous operations (shared_ptr.get() is defintely wrong!)
- conserve the namespaces: don't use many names, prefer modifiers/adjectives. write a glossary!
- avoid abbreviations, _t, etc. (gpio?)


\end{document}

